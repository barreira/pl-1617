\documentclass{article}
\usepackage[utf8]{inputenc}
\usepackage[portuguese]{babel}
\usepackage{listings}
\usepackage{xcolor}
\usepackage{indentfirst}
\usepackage{hyperref}
\setlength{\parindent}{6ex}
\usepackage{graphicx}
\lstset{basicstyle=\ttfamily,
  showstringspaces=false,
  commentstyle=\color{red},
  keywordstyle=\color{blue},
  tabsize=3
}
\title{\textbf{Processamento de Linguagens - TP1b}}
\author{\begin{tabular}{c}
            \textbf{} Carlos Pereira (A61887) \\ João Barreira (A73831) \\ Rafael Costa (A61799)
        \end{tabular}
       }
\date{Abril 2017}

\begin{document}

\maketitle

\newpage
\renewcommand*\contentsname{Índice}

\tableofcontents



\newpage
\section{Introdução}

O presente trabalho consiste no desenvolvimento de filtros de texto recorrendo à ferramenta \emph{Flex}. Os diferentes filtros devem ser produzidos com o recurso a \emph{Expressões Regulares} para detetar \emph{padrões de frases}. Para aumentar estas capacidades, foram propostos um conjunto de exercícios. Neste trabalho, optou-se por resolver o exercício seis, pois apresenta uma complexidade considerável e, para além disso, achamos interessante aumentar as capacidades da linguagem em \emph{C} visto ser bastante utilizada na nossa área.

Ao longo deste relatório explicaremos, com detalhe, a resolução deste exercício, dando ênfase às \emph{Expressões Regulares} e \emph{Ações Semânticas}, bem como eventuais estruturas e variáveis auxiliares utilizadas. No final apresentaremos também o filtro de texto desenvolvido e exemplos da sua execução.



\newpage
\section{Inline Templates em linguagem C}

Um template embedido em linguagem C contém, tipicamente, texto com várias variáveis e expressões.

De acordo com esta informação, foi desenvolvido um program em \emph{Flex} que dado um ficheiro de template, gera, em C, um programa correspondente válido.


\subsection{Expressões Regulares e Ações Semânticas}

\begin{itemize}
    \item \textbf{\^{}[a-zA-Z]+\textbackslash{}=\textbackslash{}\{\textbackslash{}\{} -- Esta expressão é usada para extrair tudo o que se encontra antes do corpo de um \emph{template} (e.g. "Fli" e "Fhtml" na \textbf{Figura 1}) presentes no template. Desta forma, imprime-se no ficheiro resultante os nomes das funções (precedidas de "char*"), abrindo um parêntesis para o início dos argumentos e iniciando o estado "TEMPLATE".
    
    \item \textbf{\textless TEMPLATE\textgreater\textbackslash{}[} -- Identifica o início da codificação dos argumentos das funções (e.g. "ele" na função "Fli" da \textbf{Figura 1}). Inicia-se o estado "PARAMS".
    
    \item \textbf{\textless TEMPLATE\textgreater\textbackslash{}\}\textbackslash{}\}} -- Identifica o fim de uma função do \emph{template}. Quando um \emph{template} termina procede-se à escrita de todas as variáveis passadas como parâmetro à função gerada. De seguida, procede-se à escrita do corpo da função associando, sempre que necessário, uma variável a uma certa instrução dessa função. No final, todas as variáveis auxialiares são igualdas a zero e o programa volta ao estado "INITIAL".
    
    \item \textbf{\textless TEMPLATE\textgreater\^{}[ \textbackslash{}t]*[a-zA-Z0-9\textless\textgreater\,\,/]+} -- Serve para extrair uma linha de um \emph{template}.
    
    \item \textbf{\textless TEMPLATE\textgreater[ \textbackslash{}t]*[a-zA-Z0-9\textless\textgreater\,\,/]+\textdollar{}} -- Serva para extrair o conteúdo de uma linha de um \emph{template} após o processamento de uma variável ou de um \emph{map}. Neste caso, verifca-se o valor da varíavel \emph{mapMode}. Se o valor de \emph{mapMode} for igual a um, então isso quer dizer que se processou um \emph{map} nesta linha. Caso contrário foi processada uma variável. Em ambos os casos armazena-se a linha de texto lida num \emph{array}. Se o valor de \emph{mapMode} for igual a zero, associa-se a linha processada ao nome da variável lida. 
    
    \item \textbf{\textless TEMPLATE\textgreater[ \textbackslash{}t]*[a-zA-Z0-9\textless\textgreater\,/]+} -- Serve para extrair uma linha de um \emph{template}.
    
    \item \textbf{\textless PARAMS\textgreater "\%"} -- Identifica o início de uma variável (e.g. "tit" na \textbf{Figura 1}). Inicia-se o estado "VARS".
    
    \item \textbf{\textless VARS\textgreater "\%]"} -- Identifica o fim da codificação de uma variável. Volta-se a iniciar o estado "TEMPLATE".
    
    \item \textbf{\textless VARS\textgreater "MAP"} -- Identifica o início da codificação de um map. Aciona-se a flag \emph{mapMode}, guarda-se a linha de início do \emph{map} em \emph{mapLine} e inicia-se o estado "MAP".
    
    \item \textbf{\textless VARS\textgreater [a-zA-Z]+} -- Serve para extrair o nome das variáveis. Um nome de uma variável apenas é guardado no \emph{array} caso não exista.
    
    \item \textbf{\textless MAP\textgreater "\%]"} -- Identifica o fim da codificação de um map. Volta-se a iniciar o estado "TEMPLATE".
    
    \item \textbf{\textless MAP\textgreater [a-zA-Z ]+} -- Serve para extrair o nome dos parâmetros de um map.
    
    \item \textbf{.\textvert{}\textbackslash{}n} -- Todos os restantes carateres (incluindo \textbackslash{}n), são escritos no ficheiro resultante.
\end{itemize}


\subsection{Estrutura de Dados Globais}

Neste exercício foi utilizado um \emph{array} de {strings} para se armazenarem todas as linhas de texto provenientes de um \emph{template}, bem como uma variável auxiliar para armazenar o seu tamanho. Com o uso desta estrutura de dados conseguiu-se, facilmente, armazenar todas as linhas de todos os \emph{templates} de um ficheiro num modo sequencial.

De modo a tratar das variáveis associadas a um \emph{template}, os nomes destas foram também guardados num \emph{array} de \emph{strings}. Foi também utilizada uma variável auxiliar que guarda do tamanho deste \emph{array}.

Tendo como recurso a biblioteca \emph{glib} usou-se uma estrutura do tipo \emph{GTree} para se efetuar a associação entre uma linha de texto qualquer e uma variável presente nessa mesma linha.

Finalmente, para o tratamento de \emph{maps} recorreu-se a três variáveis do tipo \emph{string}, para armazenar o nome da função, o tamanho da lista e o nome da lista usados num \emph{map}.


\subsection{Filtro de Texto}

\begin{lstlisting}[language=C]
%option noyywrap
%x TEMPLATE PARAMS VARS MAP

%{
	#include <glib.h>


	#define MAX_LINES 10000
	#define MAX_VARS  20


	gint comp(gconstpointer, gconstpointer);


	FILE* fp;	
	int array_size = 0;
	char* array[MAX_LINES];
	char* line = NULL;
	char* var = NULL;
	GTree* tree;

	int vars_size = 0;
	char* vars[MAX_VARS];

	int mapMode = 0;
	int mapLine = 0; 
	char* functionName = "";
	char* listLength = ""; 
	char* list = "";
%}

%%

<*>^[a-zA-Z]+\=\{\{ {
	int i = 0;

	fprintf(fp, "char* ");

	for (; i < yyleng && yytext[i] != '='; i++) {
		fprintf(fp, "%c", yytext[i]);
	}

	fprintf(fp, "(");

	BEGIN TEMPLATE;
}

<TEMPLATE>\[ {
	BEGIN PARAMS;
}

<TEMPLATE>\}\} {
	int i = 0;
	gchar* auxVar = "";

	for (; i < vars_size; i++) {
		if (i > 0) {
			fprintf(fp, ", ");
		}

		fprintf(fp, "char* ");
		fprintf(fp, "%s", vars[i]);	
	}

	if (mapMode == 1) {
		if (i > 0) {
			fprintf(fp, ", int %s,", listLength);
		}
		else {
			fprintf(fp, "int %s,", listLength);
		}

		fprintf(fp, " char* %s[]", list);		
	}

	fprintf(fp, ")\n{\n");
	fprintf(fp, "\tchar BUF[10000];\n\tint j = 0;\n");
	
	for (i = 0; i < array_size; i++) {
		if (mapMode == 1 && mapLine == i) {
			fprintf(fp, "\tj += sprintf(BUF + j, \"%s\");\n", array[i]);
			fprintf(fp,
                    "\tfor(int i = 0; i < %s; i++) {
                    \n\t\tj += sprintf(BUF + j, \"%%s\", %s(%s[i])); \n\t}\n",
                    listLength,
                    functionName,
                    list);  
			fprintf(fp, "\tj += sprintf(BUF + j, \"%s\\n\");\n", array[i + 1]);
		}
		else {
			auxVar = (gchar*)g_tree_lookup(tree, array[i]);
		
			if (auxVar != NULL && strcmp(auxVar, "") != 0) {
				fprintf(fp, 
                        "\tj += sprintf(BUF + j, \"%s\\n\", %s);\n",
                        array[i],
                        auxVar);
			}
			else if (auxVar) {
				fprintf(fp, "\tj += sprintf(BUF + j, \"%s\\n\");\n", array[i]);
			}
		}
	}

	fprintf(fp, "\treturn strdup(BUF);\n}\n");
	
	array_size = 0;
	vars_size = 0;
	mapMode = 0;
	mapLine = 0;
	
	BEGIN INITIAL;
}

<TEMPLATE>^[ \t]*[a-zA-Z0-9<> /]+ {
	if (line == NULL) {
		line = strdup(yytext);
	}
	else {
		strcat(line, yytext);
	}
}

<TEMPLATE>[ \t]*[a-zA-Z0-9<> /]+$ {
	int auxMap = 0;

	if (line != NULL && mapMode != 1) {
		strcat(line, "%s");
		strcat(line, yytext);
		array[array_size++] = strdup(line);
	}
	else if (line != NULL && mapMode == 1) {
		auxMap = 1;
		
		if (var != NULL) {
			var[0] = '\0';
		}

		array[array_size++] = strdup(line);
		array[array_size++] = strdup(yytext);
	}
	else if (line == NULL) {
		if (var != NULL) {
			var[0] = '\0';
		}

		array[array_size++] = strdup(yytext);
	}	

	if (auxMap != 1) {
		if (var == NULL) {
			g_tree_insert(tree, array[array_size - 1], "");
		}
		else {
			g_tree_insert(tree, array[array_size - 1], strdup(var));
		}
	}

	line = NULL;
}

<TEMPLATE>[ \t]*[a-zA-Z0-9<> /]+ {
	if (line == NULL) {
		line = strdup(yytext);
	}
	else {
		strcat(line, yytext);
	}
}

<PARAMS>"% " {
	BEGIN VARS;
} 

<VARS>"%]" {
	BEGIN TEMPLATE;
}

<VARS>"MAP" { 
	mapMode = 1;
	mapLine = array_size;
	BEGIN MAP;
}

<VARS>[a-zA-Z]+ {
	int i = 0;
	int existVar = 0;

	var = strdup(yytext);

	for (; i < vars_size; i++) {
		if (strcmp(var, vars[i]) == 0) {
			existVar = 1;
			break;
		}
	}

	if (existVar == 0) {
		vars[vars_size++] = strdup(var); 
	}
}

<MAP>"%]" {
	BEGIN TEMPLATE;
}

<MAP>[a-zA-Z ]+ {
	functionName = strdup(strtok(yytext, " "));   
	listLength = strdup(strtok(NULL, " "));
	list = strdup(strtok(NULL, " "));
}

.|\n { fprintf(fp, "%s", yytext);}

%%

gint comp(gconstpointer a, gconstpointer b)
{
	return (strcmp(a, b));
}


int main(int argc, char** argv)
{
	char* output;

	tree = g_tree_new(comp);

	if (argc == 2) {
		yyin = fopen(argv[1], "r");

		output = strtok(argv[1], ".");
		strcat(output, ".c");
		fp = fopen(output, "w");

		yylex();
	}

	fclose(fp);
	g_tree_destroy(tree);	

	return 0;
}

\end{lstlisting}


\newpage
\subsection{Resultado da Execução}

A partir do template que é apresentado na \textbf{Figura 1} -- e através da execução do programa em \emph{Flex} --, foi gerado o código em C correspondente, presente na \textbf{Figura 2}. 

\begin{center}
    \includegraphics[scale=0.6]{exemplo1}
    \caption{\textbf{Figura 1} - Template embebido em linguagem C}
\end{center}

\begin{center}
    \includegraphics[scale=0.5]{resultado1}
    \caption{\textbf{Figura 2} - Código C resultante}
\end{center}

\newpage
A \textbf{Figura 3} apresenta outro template de exemplo e o respetivo resultado (\textbf{Figura 4}).

\begin{center}
    \includegraphics[scale=0.6]{exemplo2}
    \newline
    \caption{\textbf{Figura 3} - Template embebido em linguagem C}
\end{center}

\begin{center}
    \includegraphics[scale=0.7]{resultado2}
    \caption{\textbf{Figura 4} - Código C resultante}
\end{center}



\newpage
\section{Conclusão}

Com a realização deste trabalho, pudemos pôr em prática os conceitos adquiridos nas aulas teórico-práticas sobre a ferramenta \emph{Flex}.

Percebemos a importância do uso das expressões regulares para descrever padrões de frases, de forma a podermos pegar num ficheiro e obtermos as informações que necessitamos.

Além disso, aprendemos também a desenvolver processadores de linguagem que nos permitiram, a partir das informações obtidas anteriormente, filtrar e modificar textos.

Assim sendo, em retrospetiva, achamos que a realização deste segundo trabalho prático foi bastante enriquecedora.

\end{document}