\documentclass{article}
\usepackage[utf8]{inputenc}
\usepackage[portuguese]{babel}
\usepackage{listings}
\usepackage{xcolor}
\usepackage{indentfirst}
\setlength{\parindent}{6ex}
\lstset{basicstyle=\ttfamily,
  showstringspaces=false,
  commentstyle=\color{red},
  keywordstyle=\color{blue},
  tabsize=3
}

\title{\textbf{Processamento de Linguagens - TP1a}}
\author{\begin{tabular}{authors}
            \textbf{} Carlos Pereira (A61887) \\ João Barreira (A73831) \\ Rafael Costa (A61799)
        \end{tabular}
       }
\date{Março 2017}

\begin{document}

\maketitle

\section{Processador de transações da Via Verde}



\section{Album Fotográfico em HTML}



\section{Autores Musicais}

Existe uma diretoria com vários ficheiros de extensão '\emph{.lyr}', que contêm a letra de canções precedida de 2 ou mais campos de meta-informação (1 por linha). Esta informção pode ser: título da canção, autor da letra, cantor, etc. A letra da música e a meta-informação estão separadas por uma linha em branco.

Foi feito um Processador de Texto com o auxílio do \emph{GAWK} para ler todos os ficheiros '\emph{.lyr}' e obter informações sobre os cantores, autores e títulos de canções.

\subsection{Total de \emph{cantores} e a lista com os nomes}

Aqui foi feito um processador que recebe ficheiros de letra de música como \emph{input} e escreve o total e os nomes de todos os cantores.

\subsubsection{Expressões Regulares}

Após a análise de alguns possíveis ficheiros de entrada, verificou-se que as linhas de informações relativas aos nomes dos cantores começam por \emph{"singer:"}, onde os nomes estão separados por espaços em branco, dois pontos, vírgulas ou por pontos e vírgulas, daí os termos utilizado como \emph{Field Separator}.

A informação que queremos retirar está em linhas que começam por \emph{"singer:"}.

\subsubsection{Ações Semânticas}

Sempre que o processador encontra um registo iniciado por \emph{"singer:"}, percorrem-se todos os campos desse registo. Para cada um destes campos, retiram-se os carateres desnecessários. Depois, caso o registo não exista no \emph{array} dos nomes dos cantores, é adicionado a esse \emph{array} e o contador de todos os cantores é incrementado.

No fim, antes de se imprimirem os resultados, o \emph{array} com os nomes dos cantores é ordenado.

\subsubsection{Estruturas de Dados Globais}

Foram utilizadas duas estruturas de dados globais: um \emph{array} \emph{singers}, em que o índice são os nomes de todos os cantores e uma variável \emph{total} que é utilizada como contador de cantores.

\subsubsection{Filtro de Texto}

\begin{lstlisting}[language=bash]
BEGIN {
	FS = " *[:;,] *"
}

/singer:/ {
	for (i = 2; i <= NF; i++) {
	    sub("[ ?()]+$", "", $i);

	    if (!($i in singers) && ($i != "")) {
	    	count++;
	    	singers[$i] = $i;
	    }
	}
}

END {
	n = asort(singers);

	for (i = 1; i <= n; i++) {
		print singers[i];
	}

	print "Total: " count;
}
\end{lstlisting}


\subsection{Todas as canções do mesmo \emph{autor}}

Fez-se um processador que recebe ficheiros de letra de música como \emph{input} e calcula o número de canções de cada autor.

\subsubsection{Expressões Regulares}

Verificou-se que as linhas de informações relativas aos nomes dos autores começam por \emph{"author:"}. Tal como no caso dos cantores, os nomes estão separados por espaços em branco, dois pontos, vírgulas ou por pontos e vírgulas, daí os termos utilizado como \emph{Field Separator}.

As informações que pretendemos são retiradas de registos que começam por \emph{"author:"}.

\subsubsection{Ações Semânticas}

Ao encontrar um registo iniciado por \emph{"author:"}, o processador percorre todos os campos desse registo. Para cada um destes campos, retiram-se os carateres desnecessários. Depois, caso o registo não seja vazio, ou seja, conhecem-se os autores da música, a posição correspondente ao nome do autor no \emph{array} de autores é incrementado. Caso contrário, incrementa-se a posição correspondente ao "Autor Desconhecido".

No fim, imprimem-se os resultados da seguinte forma: nome do autor seguido do número de músicas associadas a ele.

\subsubsection{Estruturas de Dados Globais}

Temos como estrutura global um \emph{array} \emph{authors}, em que o índice é o nome de um autor e o valor correspondente ao número de músicas associado a esse autor. 

\subsubsection{Filtro de Texto}

\begin{lstlisting}[language=bash]

BEGIN {
	FS = " *[:;,] *"
}

/author:/ {
	for (i = 2; i <= NF; i++) {
		sub("[ ?()\\t]+$", "", $i);

		if ($i != "") {
			songs[$i]++;
		}
		else {
			songs["Autor desconhecido"]++;
		}

	}
}

END {
	for (i in songs) {
		print i " - " songs[i];
	}
}

\end{lstlisting}


\subsection{Escrever o nome de cada \emph{autor} seguido do título das suas canções}

\paragraph{} O processador recebe ficheiros de letra de música como \emph{input} e escreve, para cada autor, o título das suas canções.

\subsubsection{Expressões Regulares}

Aqui, as linhas de informação relativas ao nome de uma canção começam por   \emph{"title:"} e as dos nomes dos autores começam por \emph{"author:"}.

Da mesma maneira que se fez nos exercícios anteriores, os nomes estão separados por espaços em branco, dois pontos, vírgulas ou por pontos e vírgulas, daí os termos utilizado como \emph{Field Separator}.

As informações que pretendemos são retiradas de registos que começam por \emph{"title:"} e por \emph{"author:"}.

\subsubsection{Ações Semânticas}

O processador, a cada registo que comece por \emph{"title:"}, atribui esse campo à variável global \emph{song}. Aqui fica guardado o título da canção. Depois retira os carateres que são considerados desnecessários.

Aos registos que sejam iniciados em \emph{author:}, começa a percorrê-lo a partir do seu segundo campos (visto que ser \emph{author:}). Cada um destes campos é o nome de um autor. Depois, a cada um destes que não seja vazio, retiram-se os carateres desnecessários e é adicionado à variável global \empg{songs}, na linha correspondente ao nome do autor, o  conteúdo na variável \emph{song} que se traduz no título da canção.

Caso não se tenha o nome do cantor, ou seja, o campo é vazio, adiciona-se o título da canção à linha do "Autor Desconhecido".

No fim, para cada uma das linhas da matriz \empg{songs}, imprime-se para cada linha, o seu índice e os conteúdos de cada uma das suas colunas.

\subsubsection{Estruturas de Dados Globais}

Utilizaram-se 2 variáveis globais: a variável \emph{song} que guarda o título de uma canção e uma matriz \emph{songs}. Nesta matriz, as linhas correspondem a nome de autores e as respetivas colunas correspondem aos títulos das canções associadas ao autor dessa linha.

\subsubsection{Filtro de Texto}

\begin{lstlisting}[language=bash]
BEGIN {
	FS = " *[:;,] *"
}

/title: / {
	song = $2;
	sub("\\(\\?\\)", "", song);
	sub("^[ *=]+", "", song);
}

/author:/ {
	for (i = 2; i <= NF; i++) {
		sub("[ ?()\\t]+$", "", $i);

		if ($i != "") {
			authors[$i][song] = song;
		}
		else {
			authors["Autor desconhecido"][song] = song;
		}

	}
}

END {
	for (a in authors) {
		printf("%s: ", a);

		flag = 0;		

		for (j in authors[a]) {
			if (flag == 0) {
				printf("%s", j);
			}
			else {
				printf(", %s", j);
			} 

			flag++;
		}

		printf("\n");
	}
}
\end{lstlisting}



\section{Dicionauro}

\subsubsection{Expressões Regulares}

\subsubsection{Ações Semânticas}

\subsubsection{Estruturas de Dados Globais}

\subsubsection{Filtro de Texto}

\end{document}